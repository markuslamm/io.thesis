\chapter{Architecture and Implementation}

Distributed system -> distributed collection, cloud environments, microservice architecture,
service-discovery, communication via REST, Publish(client) -> Topic <- Subscribe Logstash,

continuous, distributed, time-series "data feed", difference raw and aggregated\cite{Klepp16},
events as "immutable facts", why?
arch uses both, raw because of unknown interests in data, flink-job-index for demonstration
of SP.

%Welche Teilprobleme leiten sich aus der Zielstellung weiter ab?
%Was sind die Rahmenbedingungen für die Probleme und wie können wir diese lösen.
%
%Lösungsbeschreibung, warum so?
%
%Bauen Sie auf der Methodik aus der Einführung auf.
%Ein System bauen und dann beobachten (machen Sie wohl meistens)
%
%z.B. wenn Sie ein System bauen. Wie sieht Ihre Architektur aus? Was sind die Resultate des
%Entwurfs? Welche Alternativen gibt es zu Ihrem Entwurf? Warum haben Sie sich gerade für
%Ihre Lösung entschieden? Was sind deren Vor­ und Nachteile?
%
%Welche Prozesse unterstützt die Architektur?
%Datenquellen?
%Transformationen?
%Datenverarbeitung?
\section{Architecture}

TODO

\subsection{Infrastructure Components}

make refs to taxonomy
\subsubsection{Service-Discovery}

Registraction for CollectorClients

\subsubsection{Message-Broker}
Queueing, see Marz15

compute layer

Transport, "Event-Log", see \cite{Kreps13}
Collect the streams and make them available for consumption

\subsubsection{Indexer}

Receive messages from Kafka, route data, create ES index, why, describe context BDAA

compute layer

\subsubsection{Persistence}

ES as search index for time-series based data, easy vizualization with Kibana, why?

storage layer

\subsection{Software Components}

\subsubsection{CollectorClient}

The CollectorClient tier is our entry point for bringing data into the system...
A module to gather the event streams from data sources.

data layer

\subsubsection{CollectorManager}

Gives overview, uses Consul as service-discovery

\subsubsection{CollectorDataProcessor}

module to analyze the streams creating derived streams and persist flat data -> data transformation

analytics layer

\subsection{Technologies}

%READ
%Messaging is the art of moving message across the producer and consumer
%group. It integrates distributed applications to an overall system by providing
%message queues for communication between them. One application can publish
%its messages to the queue and the other application can asynchronously read it
%from the queue at any time. Message Broker systems persist incoming
%messages in an enhanced type of a message queue, named topic. Sending
%messages to the broker in the form of publishing to a specific topic and on the
%other hand receiving messages only for the specified topic, is called publish /
%subscribe and this therefore classifies Kafka as a publish subscribe system.

%3.3 What is a Topic in Kafka?
%Kafka provides a high-level abstraction called Topic. Users define a new Topic
%(file) for each new category of messages (documents) and the messages are
%published to a category or stream name. A topic allows the message broker to
%deliver messages to multiple independent consumers.
%Kafka has a very simple storage layout. Each partition of a topic (file)
%corresponds to a logical log. Each time a producer publishes a message to a
%topic, the broker appends the message to the last segment file. The message is
%exposed to the consumers after it is flushed.
%
%3.4 Who are Producers in Kafka?
%Producers are the clients publishing messages to a Topic. Producers are diverse
%in nature and publish varied messages to different topics. A producer can
%publish data to a topic of its choice and is responsible for choosing which
%message will be assigned to which partition within a topic. Kafka producer client
%is configured to send messages in either a synchronous or asynchronous
%fashion to the topics. The asynchronous mode allows the client to batch small
%messages into larger data chunks before sending them over the network.
%
%3.5 What is a Message Broker?
%A Message Broker is a key element of the solution. It is a dedicated component
%which decouples the source and target systems by assuming full responsibility
%for coordinating communication between all the connected nodes. The
%published messages are stored at a set of servers called Brokers. Each Kafka
%cluster consists of one or more Brokers. The partitions of a Topic are distributed
%over the Brokers of the Kafka cluster with each Broker handling data and
%requests for a share of the partitions. Each partition is replicated across a
%configurable number of Brokers for fault tolerance. The main tasks of a message
%broker are the dynamic registration of endpoints, determining the location of a
%target system, and performing the communication as well as the transformation
%of a message from one format to another.
%A consumer can subscribe to one or more topics from the brokers and consume
%the subscribed messages by pulling data from the Brokers. The Broker locates
%the requested message by searching the list and sends the data back to the
%consumer. After a consumer receives a message it computes the location of the
%next message to consume and uses it in the next pull request.
%
%3.6 Who are Consumers in Kafka?
%The term consumer describes the clients that consume messages from a Topic.
%Producers and Consumers can simultaneously write to and read from multiple
%topics. A consumer always consumes messages from a particular partition
%sequentially. If the consumer acknowledges a particular message it implies that
%the consumer has received all the messages prior to that message. This leads to
%the fact that Kafka relies on the pull-model to reach maximum performance on
%the consumption side.
%Apache Kafka does not keep track of what messages have been consumed on
%the Broker and therefore does not rely on any acknowledgments from the
%consumer. Instead, the position of a consumer is just a single integer on a
%partition which defines the offset of the next message to consume. As a side
%benefit, it permits the consumer to rewind back to an old offset and re-consume
%data. Each consumer is represented as a process and these processes are
%organized within groups called consumer groups as Kafka supports the concept
%of consumer groups. Each consumer group consists of one or more consumers
%that jointly consume a set of subscribed topics.
\section{Implementation}

Introduce software stack, why used?

\section{The "collect"-algorithm}

Java8, CPs, non-blocking streams

%Eigenschaften des Algorithmus, Komplexität, Wie gehen Sie vor?, Beschreiben Sie was wann passiert

\section{Summary}

Maybe Spring alternatives, Lagom, VertX, Play?
Maybe collector as agent, Instrumentation instead of microservice, alternatives REST, maybe (Web-)Sockets
possible secururity risk because remote JMX, firewalls

