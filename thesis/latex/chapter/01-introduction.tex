\chapter{Introduction}
\section{Motivation}
According to a survey in Germany, nine out of ten companies (89 percent) analyze large volumes of data for operational decision-making
processes using modern Big Data Analytics Architectures, where 48 percent of respondents see the greatest potential of Big Data
\cite{Bitkom14}. The analysis of continuous data streams is taking up a growing importance for companies and therefore constitutes an
important factor for business success.

Collecting, storing and analyzing system and operational data of Big Data Architectures is therefore an essential tool in order
to ensure successful operation. By analyzing execution and service data, problems can be tracked and potential sources of error
identified as early as possible.

\section{Objective}

The main goal of the thesis is the design and implementation of a software system to ingest and  store system and operational
data of Big Data Analytics Applications. It should be investigated which data is available for Apache Flink and Apache Kafka, what
data is relevant and shall be collected, how to collect from source systems and how the data can be stored in a centralized persistence
system.

\section{Structure of thesis}
