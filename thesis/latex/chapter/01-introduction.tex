\chapter{Introduction}

\section{Motivation}

In preparation of this thesis, I first got in contact with Prof. Dr. Stefan Edlich on January 29th
this year and presented an own idea for a topic of a bachelor's thesis. At this time, I was
visiting my last course at Beuth University, the software-project which is spread over two semesters
aiming to design and implement a software application in cooperation with software related companies
based in Berlin, which was Lieferando in my case, an online food order service. During this project,
I got in touch with a lot of technologies like Apache Kafka, Apache Spark, Cassandra, Elasticsearch
and Consul, all together well known to me as 'buzzwords' from technology-blogs and magazines.

Because I was interested to learn a bit more about that "big-data-streaming-thing" and especially how
to build software using stream processing frameworks, I decided my thesis to be in this big data context
and created a working title "Design And Implementation Of A 'Data Processing Pipeline'” To Transform
Continous Monititoring Data Streams". The basic idea was to aggregate data from REST RabbitMQ endpoints,
send this raw data to a stream processor and create a model which fits the monitoring domain and store
this data in a storage system which enables further data analytics.

During the following email correspondence, Prof. Dr. Stefan Edlich he suggested me to get the data
from the the streaming platforms components itself, instead of a RabbitMQ queue as my idea suggested.
So he presented one of his own topics which was quite congruent to my own idea with the given title
\textit{Design and Implementation of a Tool to Collect Execution- and Service-Data of Big Data Analytics
Applications}, which I finally choosed to be the one to work out.

This topic is located on germans biggest big data research project "Berlin Big Data Center", which
Prof. Dr. Stefan Edlich is a member of. Within the project, a program will be developed, which collects
and stores relevant data of streaming platforms like Apache Flink, Apache Kafka or Apache Spark,
with the overall aim to build a software that is able to "learn", based on the data that
will be collected by the system that is proposed in this thesis.

Apache Flink is a "new player" in the plurality of stream processing frameworks. It was
initialized by researchers of the Technische Universität Berlin, Humboldt Universität
Berlin and Hasso-Plattner Institut Potsdam in 2008 and has emerged from the research project described
above.  On the 12th of January 2015 Flink became a top level project of the Apache Foundation. In the
meantime, the development of Flink is driven by a grown community (216 contributers, 22.08.2016) and
a wide range of companies that are actively using it.

\section{Objective}

The main goal of the thesis is a working software system to ingest and store data that can be collected
from Apache Flink and Apache Kafka. It will be examined, which data is available and can be collected
at all, what data is relevant and how to collect from source systems.

Furhermore, the collected data must be stored in a persistence system to become available
for possible consumers like visualization applications, analytical processes or as a data
source for applications from the context of Machine Learning for example.

This thesis will not be a deep introduction into big data, stream processing or covers deeper
details of the internals of Apache Flink and Apache Kafka. To understand the context this
frameworks are located in, the underlying concepts will be explained only briefly.

\section{Structure of thesis}

After a short introduction to the topics and the main goals of the present thesis in this
chapter, Chapter 2  discusses basic concepts of big data, stream processing, introduces Apache Flink and
Apache Kafka as representatives of widely used stream processing frameworks. In preparation of
Chapter 3, both Representational State Transfer (REST) and the Java Management Extensions (JMX)
as possibilities of remote data access in distributed systems will be discussed.

Chapter 3 examines Apache Flink and Apache Kafka regarding to the provided data both of the systems.
The different sources for the data collection will be described, as well what data should be collected
and stored in a persistence system regarding to its relevance and data quality. According to the results
of the data analysis, the functional and non-functional requirements of the system being developed will
be introduced at the end of the chapter.

Based on the requirements elaborated in Chapter 3, Chapter 4 introduces the system architecture by
giving a detailed conceptional overview of the components involved, whereas Chapter 5 describes
implementation of the details for selected items.

In chapter 6 we'll see how to setup the technical environment for the usage of the prototype
to verify the correct functionality related to the requirements defined in Chapter 4.

The last Chapter 7 covers a conclusion and gives a resumee of the present work.