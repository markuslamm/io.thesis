\chapter{Introduction}
\section{Motivation}
In preparation of this thesis, I first got in contact with Prof. Dr. Stefan Edlich on January
29th this year and presented an own idea for a topic of a bachelor's thesis. At this time,
I was visiting my last course at Beuth University, the software-project which is spread
over two semesters aiming to design and implement a software application in cooperation
with software related companies based in Berlin, which was Lieferando in my case, an
online food order service. During this project, I got in touch with a lot of technologies
like Apache Kafka, Apache Spark, Cassandra, Elasticsearch and Consul, all together well
known to me as 'buzzwords' from technology-blogs and magazines.

Because I was interested to learn a bit more about that "big-data-streaming-thing" and
especially how to build software using Stream Processing frameworks, I decided my thesis
to be in this Big Data context and created a working title "Design And Implementation Of
A Data Processing Pipeline” To Transform Continous Monititoring Data Streams". The
basic idea was to aggregate data from REST RabbitMQ endpoints, send this raw data
to a stream processor and create a model which fits the monitoring domain and store this
data in a storage system which enables further data analytics.

During the following email correspondence, Prof. Dr. Stefan Edlich he suggested me to
fetch the data from the the streaming platforms components itself, instead of a RabbitMQ
queue as my idea suggested. So he presented one of his own topics which was quite
similar to my own idea with the given title \textit{"Design and Implementation of a Tool to
Collect Execution- and Service-Data of Big Data Analytics Applications"}, which I choosed to
be the one to work out at last.

This topic is located on germans biggest Big Data research project "Berlin Big Data Center",
which Prof. Dr. Stefan Edlich is a member of. Within the project, a program will be
developed, which collects and stores relevant data of streaming platforms like Apache
Flink, Apache Kafka or Apache Spark, with the overall aim to build a software that will be
able to "learn" based on the data that will be collected by the system that is proposed in
this thesis.

Apache Flink is a "new player" in the plurality of Stream Processing frameworks. It
was initialized by researchers of the Technische Universität Berlin, Humboldt Universität
Berlin and Hasso-Plattner Institut Potsdam in 2008 and has emerged from the research
project described above.  On the 12th of January 2015 Flink became a top level project
of the Apache Foundation. In the meantime, the development of Flink is driven by a
grown community (218 contributers, August 29th 2016, see \cite{FlinkG16}) and a wide range of
companies that are actively using it.

\section{Objective}
The main goal of the thesis is the design and implementation of a working software system
to ingest and store data that can be collected from Apache Flink and Apache Kafka and
represents the potential data providing component for the self-learning system described
above. It will be examined, which data is available and can be collected at all, what data
is relevant and how to collect from source systems.

Furhermore, the collected data must be stored in a persistence system to become available
for possible consumers like visualization applications, analytical processes or as a data
source for applications from the context of Machine Learning for example.

This thesis will not be a deep introduction into Big Data, Stream Processing or covers deeper
details of the internals of Apache Flink and Apache Kafka. To understand the context this
frameworks are located in, the underlying concepts will be explained only briefly.

\section{Structure of thesis}
After a short introduction to the topic and the main goals of the present thesis in this
chapter, \autoref{ch:basic-concepts} discusses basic concepts of Big Data, Stream Processing and introduces
Apache Flink and Apache Kafka as representatives of widely used streaming frameworks. In preparation of \autoref{ch:requirements},
both Representational State Transfer (REST) and the Java Management Extensions (JMX) as possibilities of remote data access in
distributed systems will be discussed.

\autoref{ch:requirements} examines Apache Flink and Apache Kafka regarding to the provided data
both of the systems. The different sources for the data collection will be described, as
well what data should be collected and stored in a persistence system regarding to its
relevance and data quality. According to the results of the data analysis, the functional
and non-functional requirements of the system being developed will be introduced at the
end of the chapter.

Based on the elaborated requirements, \autoref{ch:architecture} introduces the system architecture by
giving a detailed conceptional overview of the components involved, whereas \autoref{ch:implementation}
discusses implementation details for selected items.

\autoref{ch:evaluation} introduces the local test environment and gives an detailed introduction to setup
the technical environment for and the usage of the prototype to verify the correct functionality
related to the requirements defined in \autoref{ch:requirements}.

The last \autoref{ch:conclusion} covers a conclusion and gives a resumee of the present work.