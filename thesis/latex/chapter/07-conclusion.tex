\chapter{Conclusion}
\label{ch:conclusion}

The last chapter \autoref{ch:evaluation} presented the evaluation of the \textit{"Collector-Platform"} based on a multi node cluster
build with the Docker software container engine, that allows to run the required infrastructure components of the platform
on the local developer machine.

This last chapter summarizes the results of the previous chapters and discusses possible optimizations and alternatives to
the chosen approaches of frameworks and technologies.

\section{Summary}

The main goal of the thesis is the design and implementation of a working software system
to ingest and store data that can be collected from Apache Flink and Apache Kafka and
represents the potential data providing component for a the self-learning system, that will be developed within
germans biggest Big Data research project "Berlin Big Data Center".


%Zusammenfassung und Ausblick (5 Seiten)
%●
%●
%●
%Was lesen wir in diesem Kapitel?
%Und warum muss ich das (als Gutachter oder Interessent lesen)?
%Wie verknüpft sich dieser Inhalt mit dem vorhergehendem(n) Kapitel (n)?
%Zusammenfassung
%●
%●
%●
%●
%Was war die Zielstellung?
%Wie war unsere Vorgehensweise?
%Konnten wir das Problem/die Probleme lösen?
%Wichtigste Erkenntnisgewinne?
%Ausblick
%●
%●
%●
%Was würden Sie an dem Thema machen wenn Ihnen jetzt jemand die nächsten drei Jahre
%finanziert?
%Was würde Google / Oracle / IBM machen?
%Sollten wir eigentlich solche Dinge, die Sie in Ihrer Arbeit machen, auch wirklich erforschen
%oder bauen? Wer verliert dadurch, wer gewinnt?


\section{Outlook}

Unit-Test und Refactoring

Maybe Spring alternatives, Lagom, VertX, Play?

Maybe collector as agent, Instrumentation instead of separate service

Alternatives REST, maybe (Web-)Sockets

Possible secururity risk because remote JMX, firewalls and dstat process

More performance with more "system" languages, go? c?

dynamic sample collectors
